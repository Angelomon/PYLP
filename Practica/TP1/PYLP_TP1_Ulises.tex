\documentclass{article}
\renewcommand\refname{Referencias}
\renewcommand\contentsname{\'Indice de Conten\'ido}
\usepackage{graphicx}
\graphicspath{{IMG/}}
\usepackage{caption}
\usepackage{subcaption}
\usepackage{float}
\usepackage{listings}
%\lstset{basicstyle=\ttfamily,
 % showstringspaces=false,
%}

\title{\textsc{Paradigmas y Lenguajes de Programaci\'on\\Trabajo Pr\'actico N\'umero 1\\Pr\'actica}}
\author{Ulises C. Ramirez [uli.r19@gmail.com]\\H\'ector Chripczuk\\Ver\'onica Gonzalez}
\date{14 de Septiembre, 2018}

\begin{document}
\maketitle
\pagenumbering{gobble}
\newpage

\section*{C\'odigo}
Todo el c\'odigo que est\'e expresado en el documento como respuesta a alg\'un ejercicio esta contenido en la carpeta \texttt{PascalFC}, junto con el archivo \texttt{*.lst} y el correspondiente \texttt{*.obj}.\\

\section*{Versionado}
Para el corriente documento se est\'a llevando un versionado a fin de mantener un respaldo del trabajo y adem\'as proveer a la c\'atedra o a cualquier interesado la posibilidad de leer el material en la \'ultima versi\'on disponible.\\

\begin{center}
  \textsc{Repositorio}: \textit{https://github.com/ulisescolina/UC-PYLP/}
\end{center}

\hfill--\textsc{Ulises}
\tableofcontents
\pagenumbering{gobble}
\newpage

% === Inicio del Cuerpo del Documento === %
\pagenumbering{arabic}
\section{Instalaci\'on PascalFC}
\label{sec:pascalfc}
\textsc{Consigna}: \textbf{Investigar y describir como instalar el lenguaje PascalFC en su sistema operativo. Lectura recomendada por la c\'atedra: p\'agina del Ing. John Coppens, \textit{http://jcoppens.com/soft/pfc2}}.\\

\textit{Descripci\'on de la instalaci\'on}: la descripci\'on a brindar se realiza en una m\'aquina con las siguientes caracter\'isticas:\\

\begin{lstlisting}[caption={Caracter\'isticas sistema}]
~$ uname --kernel-name --kernel-release --machine
    --operating-system
Linux 4.15.0-34-generic x86_64 GNU/Linux
~$
\end{lstlisting}

para iniciar con la instalaci\'on se sigui\'o el v\'inculo a la p\'agina del Ing. John Coppens en el apartado de descargas [http://jcoppens.com/soft/pfc2/download.php], luego se procedi\'o a descargar la ultima versi\'on de la compilaci\'on del \texttt{pfc2}, que para el d\'ia 16 de Septiembre del 2018 es \texttt{pfc2-0.9.40.x86\_64.tar.gz}. Con el archivo comprimido descargado, solamente es necesario descomprimirlo en alguna carpeta que tengamos a mano, y despues de eso utilizar los dos archivos que son el resultado de la compilaci\'on del PascalFC, \texttt{pfc2} y el \texttt{pfc2int}, ah\'i tendremos el compilador e int\'erprete.\\

Finalizados estos pasos ya tendremos instalado el PascalFC, para la compilaci\'on y ejecuci\'on ser\'a necesario realizar en consola los siguientes pasos:

\begin{lstlisting}[caption={Compilaci\'on de y ejecuci\'on con pfc2}]
~$ cd /ruta/en/la/que/se/descomprimio/la/descarga/
~$ ./pfc2 archivo_a_ser_compilado.pfc
** Sucede la compilacion **
~$ ./pfc2int archivo_a_ser_compilado.obj
~$
\end{lstlisting}

\section{Codificaci\'on 1}
\textsc{Consigna}: \textbf{Realizar un programa que ejecute paralelamente 2 procesos donde cada uno imprima por pantalla un numero ``ID'' de proceso}.\\

\lstinputlisting[language=Pascal, caption={TP1, Ejercicio 2}, label=tp1ej2]{PascalFC/tp1_ej2.pas}

Una cuesti\'on a tener en cuenta en el \texttt{Listing \ref{tp1ej2}} es  el hecho de que la funci\'on \texttt{writeln} no es at\'omica, y es muy probable que se encuentre con intercalamiento aun mas de lo que ocurre con la funcion \texttt{write}.

Alivianar un poco esta situaci\'on de intercalamiento en el ejercicio, \textit{sin el uso de sem\'aforos} es imprimiendo un parametro al lado del otro utilizando la funcion \texttt{write} como se demuestra a continuaci\'on:

\lstinputlisting[language=Pascal, caption={TP1, Ejercicio 2 (write)}, label=tp1ej2_2]{PascalFC/tp1_ej2_2.pas}

\section{Codificaci\'on 2}
\textsc{Consigna}: \textbf{realizar un programa que ejecute paralelamente 3 procesos, 2 de los procesos deben imprimir 5 n\'umeros pares y el otro 10 n\'umeros pares.}

\lstinputlisting[language=Pascal, caption={TP1, Ejercicio 3}, label=tp1ej3]{PascalFC/tp1_ej3.pas}

Cabe mencionar que en el c\'odigo anterior tambi\'en se padece de un caso bastante grave de intercalaci\'on.

\section{Codificaci\'on 3}
\textsc{Consigna}: \textbf{crear un algoritmo que calcule todos los n\'umeros primos entre 1 y 100. Distribuir los datos para que cada proceso tome el mismo n\'umeor de elementos. ?`Es una distribuci\'on \'optima? Justifique.}

Antes de presentar el c\'odigo del programa se aclara que fue necesario el uso de sem\'aforos aunque el ejercicio no lo pida, para poder as\'i presentar la informaci\'on solicitada de manera legible, de otra manera no iba a ser discernible si la implementaci\'on paralela del algoritmo tuvo \'exito.

\lstinputlisting[language=Pascal, caption={TP1, Ejercicio 4}, label=tp1ej4]{PascalFC/tp1_ej4.pas}



% === Bilbiografia === %

\begin{thebibliography}{99}
	% Item 1
	\bibitem[Gort\'azar Bellas, et al, 2012]{gortazarbellas}\textsc{Gort\'azar Bellas, Francisco; Mart\'inez Unanue, Raquel; Fresno Fern\'andez, Victor}. \textit{Lenguajes de Programaci\'on y Procesadores - Cap\'itulo 3.5}. Editorial Universitaria Ramon Areces, Madrid, 2012. \textsc{ISBN: 9788499610702}.
\end{thebibliography}
\end{document}
